% !TEX TS-program = pdflatex
% !TEX encoding = UTF-8 Unicode

% This is a simple template for a LaTeX document using the "article" class.
% See "book", "report", "letter" for other types of document.

\documentclass[11pt]{article} % use larger type; default would be 10pt

\usepackage[utf8]{inputenc} % set input encoding (not needed with XeLaTeX)

%%% Examples of Article customizations
% These packages are optional, depending whether you want the features they provide.
% See the LaTeX Companion or other references for full information.

%%% PAGE DIMENSIONS
\usepackage{geometry} % to change the page dimensions
\geometry{a4paper} % or letterpaper (US) or a5paper or....
% \geometry{margin=2in} % for example, change the margins to 2 inches all round
% \geometry{landscape} % set up the page for landscape
%   read geometry.pdf for detailed page layout information

\usepackage{graphicx} % support the \includegraphics command and options

% \usepackage[parfill]{parskip} % Activate to begin paragraphs with an empty line rather than an indent

%%% PACKAGES
\usepackage{booktabs} % for much better looking tables
\usepackage{array} % for better arrays (eg matrices) in maths
\usepackage{paralist} % very flexible & customisable lists (eg. enumerate/itemize, etc.)
\usepackage{verbatim} % adds environment for commenting out blocks of text & for better verbatim
\usepackage{subfig} % make it possible to include more than one captioned figure/table in a single float
% These packages are all incorporated in the memoir class to one degree or another...

%%% HEADERS & FOOTERS
\usepackage{fancyhdr} % This should be set AFTER setting up the page geometry
\pagestyle{fancy} % options: empty , plain , fancy
\renewcommand{\headrulewidth}{0pt} % customise the layout...
\lhead{}\chead{}\rhead{}
\lfoot{}\cfoot{\thepage}\rfoot{}

%%% SECTION TITLE APPEARANCE
\usepackage{sectsty}
\allsectionsfont{\sffamily\mdseries\upshape} % (See the fntguide.pdf for font help)
% (This matches ConTeXt defaults)

%%% ToC (table of contents) APPEARANCE
\usepackage[nottoc,notlof,notlot]{tocbibind} % Put the bibliography in the ToC
\usepackage[titles,subfigure]{tocloft} % Alter the style of the Table of Contents
\renewcommand{\cftsecfont}{\rmfamily\mdseries\upshape}
\renewcommand{\cftsecpagefont}{\rmfamily\mdseries\upshape} % No bold!

%%% END Article customizations

%%% The "real" document content comes below...

\title{HIPO Functions list}
\author{Braydon Pacheco,}
%\date{} % Activate to display a given date or no date (if empty),
         % otherwise the current date is printed 

\begin{document}
\maketitle

\section{Course List}

\textbf{function:} Display Course List\\
\textbf{Inputs:} None\\
\textbf{Outputs:} List ($@Course\_Name, @Course\_Code$)\\
\textbf{Pseudocode:} 
\begin{verbatim}
Connect to the database
Query = SELECT Course_Name, Course_Code FROM Course
Parse Query
Execute Query
Close connection to the database
\end{verbatim} \\
\textbf{function:} Get Course Description\\
\textbf{Inputs:} $@Course\_Name, @Course\_Code$\\
\textbf{Outputs:} $@Units, @Description, @Facultry\_Name$\\
\textbf{Pseudocode:} 
\begin{verbatim}
Connect to the database
Query = SELECT c.Course_Name, c.Course_Code, c.Units, c.Description, c.Faculty_Name
   FROM Course AS c WHERE c.Course_Name = @Course_Name AND c.Course_Code = @Course_Code
Parse Query
Execute Query
Close connection to the database
\end{verbatim} \\
\section{Schedule}

\textbf{function:} Generate Schedule By Program\\
\textbf{Inputs:} $@UCID, @Schedule\_Name, @Program\_Name, @Faculty\_Name$\\
\textbf{Outputs:} List ($@Course\_Name, @Course\_Code, @Section\_Number, @Building, \\
  @Room Number, @Days\_Of\_Week, @Time\_of\_Day, @Professor\_Name$)\\
\textbf{Pseudocode:} 
\begin{verbatim}
Connect to the database
Query = SELECT c.Course_Name, c.Course_Code FROM Program_Requirments AS c
     WHERE c.Program_Name = @Program_Name
Parse Query
Execute Query

For each course_name and course_code {
   If (!Student already took the course) {
       Find Course_instance in database that matches name and code
       If (Course_instance does not conflict with schedule)
           Add course instance to schedule
       Else
           Look for other course instance
   }
}

Query = SELECT (*) FROM Courses_for_semester AS c 
    WHERE c.Schedule_Name = @Schedule_Name
Parse Query
Execute Query
Close Connection to the database

Display generated schedule to the user in GUI.
\end{verbatim} \\
\textbf{function:} Delete Schedule from Database\\
\textbf{Inputs:} $@UCID, @Schedule\_Name$ \\
\textbf{Outputs:} None\\
\textbf{Pseudocode:} 
\begin{verbatim}
Connect to the database

Query = DELETE FROM courses_for_semester AS c 
          WHERE c.UCID = @UCID AND c.Schedule_name = @Schedule_name

Parse Query
Execute Query
Close connection to the database
\end{verbatim} \\
\textbf{function:} Add Course to Schedule \\
\textbf{Inputs:} $@UCID, @Schedule\_Name, @Season, @Year, @Course\_Name, \\
@Course\_Code, @Section\_Number$ \\
\textbf{Outputs:} None\\
\textbf{Pseudocode:} 
\begin{verbatim}
Connect to the database
Query = INSERT INTO courses_for_semester(@Season, @Year, @Schedule_Name, @Course_Name,
    @Course_Code, @Section_Number)
Parse Query
Execute Query
Close connection to the database
\end{verbatim} \\
\textbf{function:} Remove Course from Schedule \\
\textbf{Inputs:} $@UCID, @Schedule\_Name, @Season, @Year, @Course\_Name, \\
 @Course\_Code, @Section\_Number$ \\
\textbf{Outputs:} None\\
\textbf{Pseudocode:} 
\begin{verbatim}
Connect to the database
Query = DELETE FROM courses_for_semester AS c
    WHERE c.Schedule_Name = @Schedule_Name AND c.Season = @Season AND
    c.Year = @Year AND c.Course_Name = @Course_Name AND
    c.Course_Code = @Course_Code AND c.Section_Number = @Section_Number AND
 
Parse Query
Execute Query
Close connection to the database
\end{verbatim} \\
\section{Programs}

\textbf{function:} Display Program List\\
\textbf{Inputs:} None\\
\textbf{Outputs:} List ($@Program\_Name, @Faculty\_Name, @Major, @Minor$)\\
\textbf{Pseudocode:} 
\begin{verbatim}
Connect to the database
Query = SELECT Program_Name, Faculty_Name, Major, Minor FROM Program
Parse Query
Execute Query
Close connection to the database
\end{verbatim} \\
\textbf{function:} Show program requirments\\
\textbf{Inputs:} $@Program\_Name$\\
\textbf{Outputs:} List ($@Course\_Name, @Course\_Code$)\\
\textbf{Pseudocode:} 
\begin{verbatim}
Connect to the database
Query = SELECT p.Course_Name, p.Course_Code 
    FROM Program_Requirements WHERE p.Program_Name = @Program_Name
Parse Query
Execute Query
Close connection to the database
\end{verbatim} \\
\newpage
\section{Previous Education}
\textbf{function:} Get Courses From UofC\\
\textbf{Inputs:} $@UCID$\\
\textbf{Outputs:} None\\
\textbf{Pseudocode:} 
\begin{verbatim}
Apply web-scraping algorithm on @UCID to get list of UofC courses from UofC website
to get inputs (@Course_Name, @Course_Code, @Grade, @Season, @Year)

For each course {
    Connect to the database {
    Query = INSERT INTO Student_Takes_Course(@UCID, @Course_Name, 
      @Course_Code, @Grade, @Season, @Year)
    Parse Query
    Execute Query
    Close connection to the database
    }
\end{verbatim} \\
\textbf{function:} Add UofC Course\\
\textbf{Inputs:} $@UCID, @Course\_Name, @Course\_Code, @Grade, @Season, @Year$\\
\textbf{Outputs:} None\\
\textbf{Pseudocode:} 
\begin{verbatim}
Connect to the database
Query = INSERT INTO Student_Takes_Course(@UCID, @Course\_Name, 
  @Course\_Code, @Grade, @Season, @Year)
Parse Query
Execute Query
Close connection to the database
\end{verbatim} \\
\textbf{function:} Delete UofC Course\\
\textbf{Inputs:} $@UCID, @Course\_Name, @Course\_Code, @Season, @Year$\\
\textbf{Outputs:} None\\
\textbf{Pseudocode:} 
\begin{verbatim}
Connect to the database
Query = DELETE FROM Student_Takes_Course AS c
            WHERE c.UCID = @UCID AND c.Course_Name = @Course_Name AND 
                  c.Course_Code = @Course_Code AND c.Season = @Season AND 
                  c.Year = @Year
Parse Query
Execute Query
Close connection to the database
\end{verbatim} \\
\textbf{function:} Add non-UofC Education\\
\textbf{Inputs:} $@UCID, @Previous\_Education$\\
\textbf{Outputs:} None\\
\textbf{Pseudocode:} 
\begin{verbatim}
Connect to the database
Query = INSERT INTO Previous_Education(@UCID, @Previous_Education)
Parse Query
Execute Query
Close connection to the database
\end{verbatim} \\
\textbf{function:} Display Courses Taken\\
\textbf{Inputs:} $@UCID$\\
\textbf{Outputs:} $@Course\_Name, @Course\_Code$\\
\textbf{Pseudocode:} 
\begin{verbatim}
Connect to the database
Query = SELECT c.Course_Name, c.Course_Code 
   FROM Student_Takes_Course AS c WHERE c.UCID = @UCID
Parse Query
Execute Query
Close connection to the database
\end{verbatim} \\
\end{document}
