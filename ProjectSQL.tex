% !TEX TS-program = pdflatex
% !TEX encoding = UTF-8 Unicode

% This is a simple template for a LaTeX document using the "article" class.
% See "book", "report", "letter" for other types of document.

\documentclass[12pt]{article} % use larger type; default would be 10pt
\usepackage[utf8]{inputenc}
\usepackage[T1]{fontenc}
\usepackage{ebgaramond}
\usepackage[utf8]{inputenc} % set input encoding (not needed with XeLaTeX)

%%% Examples of Article customizations
% These packages are optional, depending whether you want the features they provide.
% See the LaTeX Companion or other references for full information.

%%% PAGE DIMENSIONS
\usepackage{geometry} % to change the page dimensions
\geometry{a4paper} % or letterpaper (US) or a5paper or....
% \geometry{margin=2in} % for example, change the margins to 2 inches all round
% \geometry{landscape} % set up the page for landscape
%   read geometry.pdf for detailed page layout information

\usepackage{graphicx} % support the \includegraphics command and options

% \usepackage[parfill]{parskip} % Activate to begin paragraphs with an empty line rather than an indent

%%% PACKAGES
\usepackage{booktabs} % for much better looking tables
\usepackage{array} % for better arrays (eg matrices) in maths
\usepackage{paralist} % very flexible & customisable lists (eg. enumerate/itemize, etc.)
\usepackage{verbatim} % adds environment for commenting out blocks of text & for better verbatim
\usepackage{subfig} % make it possible to include more than one captioned figure/table in a single float
% These packages are all incorporated in the memoir class to one degree or another...

%%% HEADERS & FOOTERS
\usepackage{fancyhdr} % This should be set AFTER setting up the page geometry
\pagestyle{fancy} % options: empty , plain , fancy
\renewcommand{\headrulewidth}{0pt} % customise the layout...
\lhead{}\chead{}\rhead{}
\lfoot{}\cfoot{\thepage}\rfoot{}

%%% SECTION TITLE APPEARANCE
\usepackage{sectsty}
\allsectionsfont{\sffamily\mdseries\upshape} % (See the fntguide.pdf for font help)
% (This matches ConTeXt defaults)

%%% ToC (table of contents) APPEARANCE
\usepackage[nottoc,notlof,notlot]{tocbibind} % Put the bibliography in the ToC
\usepackage[titles,subfigure]{tocloft} % Alter the style of the Table of Contents
\renewcommand{\cftsecfont}{\rmfamily\mdseries\upshape}
\renewcommand{\cftsecpagefont}{\rmfamily\mdseries\upshape} % No bold!

%%% END Article customizations

%%% The "real" document content comes below...


%\date{} % Activate to display a given date or no date (if empty),
         % otherwise the current date is printed 

\begin{document}
The DBMS we decided to use for our implementation was mongoDB, which is a NOSQL database. We chose mongoDB because it has an easy-to-use GUI, and is easily implemented into a python/flask webserver using pymongo, which is exactly what we did. All of our code is either in the python web server “app.py” or in the html/css documents in the “templates” folder. Since we used mongoDB and pymongo for our application, we wrote our queries using the mongoDB queries language (MQL) instead of SQL. However, I will represent these queries in SQL for the implementation section of this report. The following is all queries used in our web server in SQL format: 

\section{SQL}

\textbf{function:} get\_courselist()\\
\textbf{Inputs:} None\\
\textbf{Outputs:} List ($@department\_code, @course\_number$)
\begin{verbatim}
QUERY:
   SELECT department_code, course_code FROM Course
\end{verbatim} 
\textbf{function:} find\_dep\_courses()\\
\textbf{Inputs:} @department\_code\\
\textbf{Outputs:} List ($@department\_code, @course\_number$)
\begin{verbatim}
QUERY:
    SELECT department_code, course_code 
      FROM Course AS c
        WHERE c.department_code = @department_code
\end{verbatim} 
\textbf{function:} get\_course()\\
\textbf{Inputs:} $@department\_code, @course\_number$\\
\textbf{Outputs:} $@department\_code, @department\_name, @course\_number, @course\_name, \\ 
@course\_description, @course\_hours, @prerequisites, @antirequisites, @corequisites, @notes$
\begin{verbatim}
QUERY:
    SELECT c.department_code, c.department_name, c.course_number, c.course_name, 
          c.course_description, c.course_hours, c.prerequisites, 
          c.antirequisites, c.corequisites, c.notes
      FROM Course AS c 
        WHERE c.department_name = @department_name AND 
              c.course_number = @course_number
\end{verbatim} 
\textbf{function:} get\_programlist()\\
\textbf{Inputs:} None\\
\textbf{Outputs:} List ($@faculty\_name, @program\_name$)
\begin{verbatim}
QUERY:
   SELECT faculty_name, program_name FROM Program
\end{verbatim} 
\textbf{function:} get\_faculty\_programs()\\
\textbf{Inputs:} $@faculty\_name$\\
\textbf{Outputs:} List ($@faculty\_name, @program\_name$)
\begin{verbatim}
QUERY:
   SELECT faculty_name, program_name 
     FROM Program AS p
       WHERE @faculty_name = p.faculty_name
\end{verbatim} 
\textbf{function:} get\_program\_accreditations()\\
\textbf{Inputs:} $@faculty\_name, @program\_name$\\
\textbf{Outputs:} $@faculty\_name, @program\_name, @accreditation, @core\_courses, \\
        @outside\_faculty\_courses, @outside\_major\_courses, @option\_courses, @alternative\_courses$
\begin{verbatim}
QUERY:
   SELECT faculty_name, program_name, accreditation, core_courses, 
          outside_faculty_courses, outside_major_courses, option_courses, 
          alternative_courses
     FROM Program AS p
       WHERE p.faculty_name = @faculty_name AND p.program_name = @program_name
\end{verbatim} 
\textbf{function:} add\_previous\_education()\\
\textbf{Inputs:} $@ucid, @previous\_education$\\
\textbf{Outputs:} none
\begin{verbatim}
QUERY:
   INSERT INTO PreviousEducation (@ucid, @previous_education)
\end{verbatim} 
\textbf{function:} add\_uofc\_course()\\
\textbf{Inputs:} $@ucid, @department\_code, @course\_code, @grade, @season, @year$\\
\textbf{Outputs:} none
\begin{verbatim}
QUERY:
   INSERT INTO StudentTakesCourse (@ucid, @department_code, 
               @course_code, @grade, @season, @year)
\end{verbatim} 
\textbf{function:} display\_past\_uofc\_courses()\\
\textbf{Inputs:} $@ucid$\\
\textbf{Outputs:} List($@department\_code, @course\_code, @grade, @season, @year$)
\begin{verbatim}
QUERY:
   SELECT c.department_code, c.course_code, c.grade, c.season, c.year
     FROM StudentTakesCourse as c
       WHERE c.ucid = @ucid
\end{verbatim}
\textbf{function:} delete\_course()\\
\textbf{Inputs:} $@ucid, @department\_code, @course\_code$\\
\textbf{Outputs:} none
\begin{verbatim}
QUERY:
   DELETE FROM StudentTakesCourse AS c
     WHERE c.ucid = @ucid AND c.department_code = @department_code AND 
           c.course_code = @course_code
\end{verbatim}  
\textbf{function:} make\_schedule()\\
\textbf{Inputs:} $@ucid, @schedule\_name$\\
\textbf{Outputs:} $@ucid, @schedule\_name$\\
\begin{verbatim}
QUERY:
   INSERT INTO Schedule (@ucid, @schedule_name)
\end{verbatim}  
\textbf{function:} display\_schedule()\\
\textbf{Inputs:} $@ucid, @schedule\_name$\\
\textbf{Outputs:} $List (@department_code, @course_code, \\
@section_number, @season, @year)$\\
\begin{verbatim}
QUERY:
  SELECT s.department_code, s.course_code, s.section_number
           s.season, s.year
   FROM CoursesForSemester AS s
     WHERE s.ucid = @ucid AND s.schedule_name = @schedule_name
\end{verbatim}  
\textbf{function:} add\_course\_to\_schedule()\\
\textbf{Inputs:} $@department\_code, @course\_code, @section\_number, \\
@season, @year, @ucid, @schedule\_name$\\
\textbf{Outputs:} none
\begin{verbatim}
QUERY:
   INSERT INTO CoursesForSemester (@ucid, @schedule_name, @department_code,
                      @course_code, @section_number, @season, @year
\end{verbatim}  
\textbf{function:} delete\_course\_from\_schedule()\\
\textbf{Inputs:} $@department\_code, @course\_code, @section\_number, \\
@ucid, @schedule\_name$\\
\textbf{Outputs:} none
\begin{verbatim}
QUERY:
   DELETE FROM CoursesForSemester (@ucid, @schedule_name, @department_code,
                                   @course_code, @section_number)
\end{verbatim}  
\end{document}
